\documentclass[a4paper,10pt]{article}
\usepackage[T1]{fontenc}
\usepackage[utf8]{inputenc}
\usepackage{graphicx}
\usepackage{geometry}
\geometry{a4paper, margin=1in}
\usepackage{enumitem}
\usepackage{parskip}
\usepackage{titlesec}
\usepackage{xcolor}
\usepackage{tcolorbox}
\usepackage{fontawesome} % For icons

\titleformat{\section}
  {\normalfont\scshape}{\thesection}{1em}{}

% Set up font and spacing
\renewcommand{\familydefault}{\sfdefault}

% Define alternating colors
\newtcolorbox{altbox}[1][]{colback=#1, colframe=white, width=\textwidth, arc=0mm, boxrule=0mm, left=1mm, right=1mm, top=1.5mm, bottom=1.5mm}

% Variable to track if we should use gray or white
\newcounter{boxnumber}
\newcommand{\setboxcolor}{%
  \stepcounter{boxnumber}%
  \ifodd\value{boxnumber}
    \def\boxcolor{gray!10}%
  \else
    \def\boxcolor{white}%
  \fi
}

% Document starts
\begin{document}

% Title and contact
\begin{center}
    \huge \textbf{Thommy Westlund}\\
    \vspace{0.2cm}
    \Large UX/UI-Ingenjör\\
    \vspace{0.3cm}
    \normalsize Thommy.westlund@gmail.com \quad | \quad 070-49 45 036 \quad | \quad Malmö, Sverige
\end{center}

\vspace{0.5cm}

% Introduction
\section*{Sammanfattning}
\vspace{0.2cm}

\noindent
“Mer än 9 års erfarenhet som systemutvecklare i olika roller, från fullstack- och frontendutvecklare till UX-designer och projektledare. Jag trivs bäst med människor, vilket är anledning till dragningen mot användarcentrerade roller. Min tekniska bakgrund gör mig perfekt som länken mellan verksamhet och teknik.”
\vspace{0.3cm}

% Skills section
\section*{Färdigheter}

\textbf{UX/UI:}\\
User stories, Interviews, User testing, Sketches/Prototypes, Workshops, A/B testing, Figma, Adobe XD, Axure, Sketch, Photoshop.

\vspace{0.3cm}

\textbf{Teknik:}\\
C\#, .Net Core, JavaScript, Vue, TypeScript, React, Svelte, CSS, SCSS, AWS, Azure, Git, UML.

\vspace{0.3cm}

\textbf{Utveckling:}\\
Project management, Jira, Agile development, DevOps, Scrum.

\vspace{0.3cm}

\textbf{Språk:}\\
Svenska (Modersmål), Engelska (Flytande), Danska (Bra).

\vspace{0.5cm}

% Experience section
\section*{Exempel på uppdrag}

\textbf{Kammarkollegiet - UX/UI-Designer}\\
Kammarkollegiet hade ett behov av att utveckla ett system för att ge arbetsgivare omställningsstöd. Thommys uppgift var att analysera förstudier och processkartläggningar som hade tagits fram av kravställare. Efter att ha gått igenom detta material skapade Thommy skisser och prototyper i Figma för att visualisera systemlösningar. 

Prototyperna blev en central del av den fortsatta utvecklingen och säkerställde att arbetsgivarnas behov blev korrekt representerade i systemet.

\vspace{0.3cm}

\textbf{Jordbruksverket - UX/UI-Designer}\\
Jordbruksverket behövde omedelbart ersätta två UX-designers och en testare i ett kritiskt projekt. Thommy deltog i kravarbete, användarintervjuer och user story-mapping för att förstå behoven för ett specifikt system. Han producerade skisser och prototyper i Figma och stod ansvarig expert kring tillgänglighetsfrågor.

Thommys insats bidrog till en smidig övergång och snabb implementation av nya lösningar, samtidigt som tillgänglighetsaspekter säkerställdes i flera projekt.

\vspace{0.3cm}

\textbf{Hoodin AB - UX-Ingenjör}\\
Hoodin behövde ersätta två frontend-utvecklare och samtidigt driva nyutveckling inom företagets plattform. Thommy tog ansvar för behovs- och kravanalys genom användarintervjuer och producerade mockups i Sketch/Figma, samt utförde frontend-utveckling. Han ledde ett behovsstyrt arbete baserat på user stories och personas, vilket bidrog till skapandet av nya funktioner och förbättrad kodbas.

\vspace{0.3cm}

Hoodin vann flera viktiga affärer tack vare den utvecklade funktionaliteten, och huvudapplikationen uppgraderades framgångsrikt till Vue 3 och TypeScript, med en mer effektiv och lättförståelig kodbas.

\vspace{0.5cm}

% Experience indept section
\section*{Tidigare projekt och uppdrag}

\textbf{Mercedes-Benz Finans Sverige AB och Danmark AS - Teknisk Produktägare}\\
\normalsize \faCalendar \ Aug. 2024 - Pågående \quad \faMapMarker \ Malmö, Sverige

Mercedes-Benz Finans hade behov av en tekniskt skicklig produktägare för at förbättra interna processer bäde inom systemutveckling och för användarna av deras kredithanteringssystem, CPMS.

Thommy var ansvarig beställare gentemot leverantörer och som expert för systemet CPMS. Han fick i uppdrag att förbättra dokumentation, strukturera backloggen och säkerställa att arbetet följde en agil arbetsmetodik, Scrum.

Thommy tog ledningen för att strukturera och effektivisera tvecklingsarbetet genom att införa bättre processer för backloghantering och dokumentation. Han arbetade nära bäde kollegor och leverantörer för att skapa tydlighet ikrav och leveranser, samtidigt som han sag till att arbetet utfördes enligt de agila principerna i Scrum.

Genom Thommys insatser förbättrades bade processerna kring systemutveckling och användarupplevelsen av CPMS. Han blev uppskattad av kollegor, leverantörer och anändare för sitt engagemang och förmaga att leverera lösningar som mötte verksamhetens och anändarnas behov.

\vspace{0.5cm}
\textbf{Örnsköldsviks Hamn och Logistik AB - UX/UI-Designer}\\
\normalsize \faCalendar \ Dec. 2023 - Juni 2024 \quad \faMapMarker \ Örnsköldsvik (distans), Sverige

Örnsköldsviks Hamn och Logistik AB hade behov av att utveckla en ny externwebb och intranät som skulle baseras på deras befintliga design och möjligheterna inom CMS-systemet Sitevision.

Thommy var ansvarig för att förvalta de krav som ställts genom dokumentation och intervjuer. Han skapade skisser och mockups som kontinuerligt testades och presenterades för beslutsfattare. Dessutom tog han fram en unik grafisk profil baserat på den befintliga externa webbdesignen och möjligheterna i Sitevision.

Thommy arbetade tätt med beställare och kollegor för att säkerställa att alla designlösningar var i linje med deras behov och tekniska möjligheter. Genom att vara lyhörd och med god förståelse för Sitevision, kunde han utveckla lösningar som var både effektiva och enkla att implementera i systemet.

Både beställare och kollegor var mycket nöjda med de underlag Thommy tog fram, då han visade stor förståelse för deras behov och levererade lösningar som både var estetiskt tilltalande och tekniskt genomförbara i Sitevision.

\vspace{0.5cm}
\textbf{Örnsköldsviks Kommun - UX/UI-Designer}\\
\normalsize \faCalendar \ Dec. 2023 - Juni 2024 \quad \faMapMarker \ Örnsköldsvik (distans), Sverige

Örnsköldsviks Kommun hade behov av att utveckla ett nytt intranät som skulle vara unikt och anpassat efter de kommunalt anställdas behov, men med en begränsad budget.

Thommy ansvarade för att leda djupgående workshops för att identifiera kommunens behov och skapa ett brett designunderlag med olika prisklasser. Han samlade in krav genom dokumentation och intervjuer för att utveckla ett hållbart och användarvänligt gränssnitt som tillgodosåg de kommunalt anställdas behov.

Genom nära samarbete med olika beställare, som hade varierande krav på tillgänglighet och design, utvecklade Thommy ett gränssnitt som balanserade dessa behov. Han tog fram flera designförslag som passade inom kommunens budget och arbetade för att säkerställa att lösningarna var långsiktigt hållbara och effektiva.

Thommy lyckades tillfredsställa beställarnas olika behov genom att skapa ett intranät som både var kostnadseffektivt och funktionellt. Detta resulterade i en lösning som mötte kommunens höga krav på design och tillgänglighet inom de givna budgetramarna.

\vspace{0.5cm}
\textbf{Jordbruksverket - UX/UI-Designer}\\
\normalsize \faCalendar \ Mars 2023 - Nov. 2023 \quad \faMapMarker \ Jönköping, Sverige

Jordbruksverket hade ett akut behov av att ersätta två UX-designers och en testare inom fyra kritiska projekt. Projekten krävde snabb och effektiv implementering av lösningar för att möta användarnas behov och säkerställa tillgänglighet enligt WCAG-standarder.

Thommy blev ansvarig för att säkerställa att kraven för varje projekt tydligt definierades genom user story-mapping och användarintervjuer. Han skapade detaljerade skisser och interaktiva prototyper i Axure, Adobe XD och Figma, beroende på projektets tekniska krav. Dessutom var han expert på tillgänglighet och genomförde omfattande tester för webben samt Android och iOS.

Thommy implementerade ett agilt arbetssätt och arbetade nära både utvecklingsteamet och användarna för att kontinuerligt förbättra lösningarna. Han ledde också workshops för att öka teamets förståelse för UX och tillgänglighet, samt för att säkerställa att rätt verktyg och metoder användes. Med sin tekniska bakgrund bidrog Thommy även till att stödja frontend-utvecklare i utformandet av hållbara lösningar för HTML-semantik och tillgänglighet.

Tack vare Thommys insatser kunde Jordbruksverket snabbt och effektivt fortsätta sina projekt utan avbrott, samtidigt som användarupplevelsen och tillgängligheten förbättrades avsevärt. Hans arbete uppskattades särskilt för den snabba överlämningen och den höga kvaliteten på de lösningar som levererades. En viktig lärdom från dessa projekt var att tidigt och tydligt diskutera förväntningarna på UX-rollen, då synen på UX varierade mellan teamen på Jordbruksverket, vilket förbättrade samarbetet och resultatet överlag.

\vspace{0.5cm}
\textbf{Lösningsarkitekt - Lösningsarkitekt}\\
\normalsize \faCalendar \ Dec. 2022 - Mars 2023 \quad \faMapMarker \ USA (distans), Sverige

Teledyne FLIR anlitade Thommy för ett kortare projekt där de behövde utvärdera hur de skulle hantera en viktig applikation som då var baserad på en on-prem-lösning inhouse.

Thommys uppdrag var att genomföra en grundlig utvärdering och presentera en initiell utförandeplan med två möjliga vägval: att behålla eller byta ut applikationen, samt vägen mot en molnbaserad lösning.

Thommy genomförde intervjuer med nyckelpersoner inom bolaget för att förstå deras behov och prioriteringar. Han utvärderade även en existerande tredjepartslösning som ett potentiellt alternativ till den nuvarande applikationen. Genom detta arbete tog Thommy fram en genomförbar plan för båda vägvalen.

Thommy levererade en detaljerad utförandeplan baserat på de två alternativen, vilket gav Teledyne FLIR en tydlig väg framåt. Hans samarbetsvillighet och lösningsorienterade arbetssätt uppskattades av alla inblandade, vilket bidrog till ett positivt slutresultat.

\vspace{0.5cm}
\textbf{Kammarkollegiet - UX/UI-Designer}\\
\normalsize \faCalendar \ Dec. 2022 - Juni 2023 \quad \faMapMarker \ Karlstad (distans), Sverige

Kammarkollegiet hade behov av att utveckla ett system för omställningsstöd. Deras förstudie var redan framtagen, men de behövde en digital lösning som effektivt uppfyllde verksamhetens och användarnas behov.

Thommys uppdrag var att analysera det befintliga förstudiematerialet och processkartläggningarna. Därefter skulle han producera skisser och prototyper i Figma, samt agera sakkunnig i leveransen.

Thommy tog ledningen i att vidare specificera och fördjupa processerna genom att skapa fullständiga processkartläggningar. Han genomförde även användarintervjuer för att gemensamt med dem utveckla nya processer. Genom kontinuerlig dialog med beställarna kunde han säkerställa att både krav och design var i linje med förväntningarna.

Projektet resulterade i en fullt fungerande och användarvänlig lösning som uppfyllde Kammarkollegiets behov. Det blev ett framgångsrikt och givande projekt där Thommy fick chansen att fokusera på UX-design utan att vara direkt delaktig i den tekniska implementeringen, vilket var en lärorik omställning.

\vspace{0.5cm}
\textbf{Hoodin AB - UX-Ingenjör}\\
\normalsize \faCalendar \ Feb. 2021 - Dec. 2022 \quad \faMapMarker \ Malmö, Sverige

Thommy blev anställd som frontendutvecklare på Hoodin med initialt ansvar för AWS-infrastruktur. Med tiden utökades hans roll till att omfatta kravställning baserat på efterfrågan från både existerande och potentiella kunder. Samtidigt arbetade Thommy med att vidareutveckla företagets egna produkter och förvaltade även externa webbprojekt.

Thommy var ansvarig för att utveckla frontend-applikationer i Vue.js, där han också ledde uppgraderingen från Vue 2 till Vue 3. Hans arbete innefattade både JavaScript och TypeScript, vilket bidrog till en mer robust och flexibel kodbas.

Thommy tog över ledningen för kravanalys, specifikation och UX-frågor, baserat på marknadsbehov och användarintervjuer. Han arbetade i nära samarbete med teamet för att säkerställa att produkterna och lösningarna levererade hög kundnytta. Han var också aktiv i förvaltningen och vidareutvecklingen av webbplatser för vandrings- och cykelleder i södra Sverige, som inkluderade projekt för Region Skåne, Halland och flera kommuner.

Genom Thommys insatser förbättrades produkternas prestanda och användarupplevelse, vilket ledde till ökad kundnöjdhet och nya affärsmöjligheter för Hoodin. Uppgraderingen till Vue 3 och en effektivare kodbas gjorde systemet lättare att underhålla och utveckla vidare, vilket underlättade för framtida skalbarhet och utveckling.

\vspace{0.5cm}
\textbf{Inwido AB - UX/UI-Designer \& Fullstackutvecklare}\\
\normalsize \faCalendar \ Nov. 2016 - Feb. 2021 \quad \faMapMarker \ Jönköping, Sverige

Thommy hade en nyckelroll i både utvecklingen och förvaltningen av Inwidos IT-system och integrationslösningar, med ett särskilt fokus på att förbättra användarupplevelse och effektivitet. I rollen som systemutvecklare ansvarade han för kravhantering, kundkontakt och utveckling av funktioner i det affärskritiska systemet Konfiguratorn, som användes av återförsäljare (B2B) för att hantera orderläggning. Han tog fram prototyper i Sketch, genomförde användartester och AB-tester, samt initierade en plattformsförnyelse där systemet migrerades från Webforms/.NET till React med .NET Core, vilket resulterade i en mer skalbar och effektiv lösning.

Thommy var också ansvarig för utvecklingen av applikationen “Bildritaren,” en 3D-renderare som visualiserade fönsterkonfigurationer baserat på maskinkod från produktionen. Genom sin systemtolkning i C\# säkerställde Thommy att applikationen kunde hantera miljontals möjliga fönsterkombinationer och exakt placera spröjs och färger. Detta ledde till en kraftig minskning av fel och reklamationer, vilket förbättrade både produktkvaliteten och kundnöjdheten.

Utöver dessa utvecklingsprojekt designade och implementerade Thommy en C\# REST-baserad integrationslösning mellan Inwidos verksamhetskritiska system och en ny mobilapplikation. Genom att införa en ORM-lösning och använda LINQ för databashantering förbättrade han effektiviteten i datakommunikationen och lade grunden för framtida användning av Entity Framework.

Genom sina insatser bidrog Thommy till att modernisera och förbättra Inwidos system, vilket ökade effektiviteten och skalbarheten, samtidigt som användarupplevelsen och kundnöjdheten förbättrades.

\vspace{0.5cm}
\textbf{NYCE Solutions AB - UX/UI-Designer \& Fullstackutvecklare}\\
\normalsize \faCalendar \ Sept. 2015 - Sept. 2016 \quad \faMapMarker \ Jönköping, Sverige

Thommy ansvarade för utveckling av e-handelsplattformar och integrationer med WMS-systemet NYCE Logic, som användes för lagerhantering och orderhantering av företag som DanX och ColliCare.


Thommy var ansvarig för kundkontakt, kravhantering och utveckling av ny funktionalitet för e-handelsplatt-
formar i .NET/C\#. Han arbetade med att vidareutveckla systemet genom användarintervjuer, prototyper i Sketch och genomförde AB-testning för att säkerställa att plattformarna mötte kundernas behov.

Thommy deltog aktivt i workshops och uppsättningar av lager för att optimera processerna, samt tog fram förslag på nyutveckling baserat på kundernas behov. Han säkerställde att alla tekniska lösningar och integrationer med WMS-systemet implementerades smidigt och effektivt.

Genom Thommys insatser förbättrades både funktionaliteten och användarupplevelsen på e-handelsplatt-
formarna, vilket resulterade i nöjdare kunder och effektivare systemintegrationer. Hans deltagande i workshops hos företag som Kjell \& Company och Bring bidrog till att utveckla nya innovativa lösningar för dessa plattformar.

\vspace{0.5cm}

\section*{Kompetenser}

% Section for Expert level
\textbf{5. Expert inom området}

User Story Mapping, UX Design, UX/UI Design, Accessibility (WCAG), System Development, Availibility Testing, HTML, CSS3, Prototyping (Low-Fidelity/High-Fidelity), User Testing and Analysis, User Research, Agile Development (Scrum, Kanban), Process Mapping, Accessibility Testing (Web, Android, iOS), Figma, Adobe Suite (Illustrator, Photoshop, XD), Sketch, HTML5, SCSS/SASS, UML, User-Centered Design \& Product Development, Requirements Analysis, Wireframes / Mockups, Test of Usability (A/B Testing), Microsoft Visio, Responsive Design, Visual Studio Code, User Scenarios, Analysis and Requirements Gathering, Agile Scrum, Wireframing, Persona Development, Journey Mapping, Flowcharting, Interaction Design
\vspace{0.3cm}

% Section for Very High level
\textbf{4. Mycket hög kompetens}

Vue 3, Vue.js, ASP.NET MVC, Entity Framework, VoiceOver, TalkBack, RESTful APIs, JIRA, Trello, Miro, JavaScript, TypeScript, React.js, .Net, .Net Core, C\#, UI Design, Frontend Development, Design Systems, Git, GitHub, DevOps, MySQL, MS SQL, Axure, Web Forms, Design Thinking, draw.io, RESTful API, Project Management, SOAP API, Microsoft SQL Server, Workshops, Material Design, Windows, Business Development, Cross-Browser Compatibility, Version Control (Git), ESLint, Color Theory, Generative AI
\vspace{0.5cm}

% Section for High level
\textbf{3. Hög kompetens}

SVN Tortoise, NPM, Webpack, Three.js, Bootstrap, AWS, Microsoft Azure, Nuxt.js, Angular, Node.js, Express.js, Visual Studio, Tailwind CSS, Content Management Systems, CMS, LINQ, Copywriting, AJAX, JQuery, Android Studio, MAC, Integration, NPM/Yarn, Domain-Driven Development, E-commerce Platforms, Information Architecture, Microinteractions, Sitevision
\vspace{0.5cm}

% Section for Medium level
\textbf{2. Medelkompetens}

Pyramid, PHP 7, SQL Server Management Studio, SAFe (Scaled Agile Framework), Python, Manufacturing Industry, Healthcare Industry, Window Manufacturing, Babel, Logistics \& Warehouse Management Systems (WMS), Microsoft Internet Information Services (IIS), Cypress, Jest, InVision
\vspace{0.5cm}

% Section for Basic level
\textbf{1. Grundläggande kompetens}

Terraform, Linux, Entertainment Industry, GraphQL

\vspace{0.5cm}

\section*{Certifikat}

% Set color and draw box
\setboxcolor
\begin{altbox}[\boxcolor]
    \textbf{Certified Professional in Accessibility Core Competencies} \hfill 
    
    \faCalendar \ Mars 2024 - Mars 2027 \quad \faMapMarker \ International Association of Accessibility Professionals
\end{altbox}

% Education section
\section*{Utbildning}
\setcounter{boxnumber}{0}
% Set color and draw box
\setboxcolor
\begin{altbox}[\boxcolor]
    \textbf{Inbyggda System} \hfill 
    
    \faCalendar \ Aug. 2013 - Juni 2015 \quad \faMapMarker \ Jönköpings Tekniska Högskola, Jönköping
\end{altbox}
\vspace{-5pt}
\setboxcolor
\begin{altbox}[\boxcolor]
    \textbf{Webbutvecklare} \hfill 
    
    \faCalendar \ Aug. 2012 - Juni 2013 \quad \faMapMarker \ Högskolan Väst, Trollhättan
\end{altbox}

\vspace{0.8cm}

% Work experience section
\section*{Arbetsgivare}

\setboxcolor
\begin{altbox}[\boxcolor]
    \textbf{Mercedes-Benz Finans Sverige AB} \hfill 
    
    \faCalendar \ Aug. 2024 - Pågående \quad \faMapMarker \ Malmö, Sverige
\end{altbox}
\vspace{-5pt}
\setboxcolor
\begin{altbox}[\boxcolor]
    \textbf{B3 Grit AB} \hfill 
    
    \faCalendar \ Dec. 2022 - Juni 2024 \quad \faMapMarker \ Malmö, Sverige
\end{altbox}
\vspace{-5pt}
\setboxcolor
\begin{altbox}[\boxcolor]
    \textbf{Hoodin AB} \hfill 
    
    \faCalendar \ Feb. 2021 - Dec. 2022 \quad \faMapMarker \ Malmö, Sverige
\end{altbox}
\vspace{-5pt}
\setboxcolor
\begin{altbox}[\boxcolor]
    \textbf{Consid AB} \hfill 
    
    \faCalendar \ Nov. 2016 - Feb. 2021 \quad \faMapMarker \ Jönköping, Sverige
\end{altbox}
\vspace{-5pt}
\setboxcolor
\begin{altbox}[\boxcolor]
    \textbf{NYCE Solutions AB} \hfill 
    
    \faCalendar \ Sept. 2015 - Aug. 2016 \quad \faMapMarker \ Malmö, Sverige
\end{altbox}

\vspace{0.8cm}
% Footer
\begin{center}
    \textit{För mer information, besök min LinkedIn:}\\
    \texttt{https://www.linkedin.com/in/thommy-westlund-02090650/}
\end{center}

\end{document}