\documentclass[a4paper,10pt]{article}
\usepackage[T1]{fontenc}
\usepackage[utf8]{inputenc}
\usepackage{graphicx}
\usepackage{geometry}
\geometry{a4paper, margin=1in}
\usepackage{enumitem}
\usepackage{parskip}
\usepackage{titlesec}
\usepackage{xcolor}
\usepackage{tcolorbox}
\usepackage{fontawesome} % For icons

\titleformat{\section}
  {\normalfont\scshape}{\thesection}{1em}{}

% Set up font and spacing
\renewcommand{\familydefault}{\sfdefault}

% Define alternating colors
\newtcolorbox{altbox}[1][]{colback=#1, colframe=white, width=\textwidth, arc=0mm, boxrule=0mm, left=1mm, right=1mm, top=1.5mm, bottom=1.5mm}

% Variable to track if we should use gray or white
\newcounter{boxnumber}
\newcommand{\setboxcolor}{%
  \stepcounter{boxnumber}%
  \ifodd\value{boxnumber}
    \def\boxcolor{gray!10}%
  \else
    \def\boxcolor{white}%
  \fi
}

% Document starts
\begin{document}

% Title and contact
\begin{center}
    \huge \textbf{Thommy Westlund}\\
    \vspace{0.2cm}
    \Large UX/UI-Ingenjör\\
    \vspace{0.3cm}
    \normalsize Thommy.westlund@gmail.com \quad | \quad 070-49 45 036 \quad | \quad Malmö, Sverige
\end{center}

\vspace{0.5cm}

% Introduction
\section*{Sammanfattning}
\vspace{0.2cm}

\noindent
“Mer än 9 års erfarenhet som systemutvecklare i olika roller, från fullstack- och frontendutvecklare till UX-designer och projektledare. Jag trivs bäst med människor, vilket är anledning till dragningen mot användarcentrerade roller. Min tekniska bakgrund gör mig perfekt som länken mellan verksamhet och teknik.”
\vspace{0.3cm}

% Skills section
\section*{Färdigheter}

\textbf{UX/UI:}\\
User stories, Interviews, User testing, Sketches/Prototypes, Workshops, A/B testing, Figma, Adobe XD, Axure, Sketch, Photoshop.

\vspace{0.3cm}

\textbf{Teknik:}\\
C\#, .Net Core, JavaScript, Vue, TypeScript, React, Svelte, CSS, SCSS, AWS, Azure, Git, UML.

\vspace{0.3cm}

\textbf{Utveckling:}\\
Project management, Jira, Agile development, DevOps, Scrum.

\vspace{0.3cm}

\textbf{Språk:}\\
Svenska (Modersmål), Engelska (Flytande), Danska (Bra).

\vspace{0.5cm}

% Experience section
\section*{Exempel på uppdrag}

\textbf{Kammarkollegiet - UX/UI-Designer}\\
Kammarkollegiet hade ett behov av att utveckla ett system för att ge arbetsgivare omställningsstöd. Min uppgift var att analysera förstudier och processkartläggningar som hade tagits fram av kravställare. Efter att ha gått igenom detta material skapade jag skisser och prototyper i Figma för att visualisera systemlösningar. 

Prototyperna blev en central del av den fortsatta utvecklingen och säkerställde att arbetsgivarnas behov blev korrekt representerade i systemet.

\vspace{0.3cm}

\textbf{Jordbruksverket - UX/UI-Designer}\\
Jordbruksverket behövde omedelbart ersätta två UX-designers och en testare i ett kritiskt projekt. Jag deltog i kravarbete, användarintervjuer och user story-mapping för att förstå behoven för ett specifikt system. Han producerade skisser och prototyper i Figma och stod ansvarig expert kring tillgänglighetsfrågor.

Min insats bidrog till en smidig övergång och snabb implementation av nya lösningar, samtidigt som tillgänglighetsaspekter säkerställdes i flera projekt.

\vspace{0.3cm}

\textbf{Hoodin AB - UX-Ingenjör}\\
Hoodin behövde ersätta två frontend-utvecklare och samtidigt driva nyutveckling inom företagets plattform. Jag tog ansvar för behovs- och kravanalys genom användarintervjuer och producerade mockups i Sketch/Figma, samt utförde frontend-utveckling. Han ledde ett behovsstyrt arbete baserat på user stories och personas, vilket bidrog till skapandet av nya funktioner och förbättrad kodbas.

\vspace{0.3cm}

Hoodin vann flera viktiga affärer tack vare den utvecklade funktionaliteten, och huvudapplikationen uppgraderades framgångsrikt till Vue 3 och TypeScript, med en mer effektiv och lättförståelig kodbas.

\vspace{0.5cm}

% Experience indept section
\section*{Tidigare projekt och uppdrag}

\textbf{Mercedes-Benz Finans Sverige AB och Danmark AS - Teknisk Produktägare}\\
\normalsize \faCalendar \ Aug. 2024 - Pågående \quad \faMapMarker \ Malmö, Sverige

Som teknisk produktägare på Mercedes-Benz Finans förbättrar jag interna processer för både systemutveckling och användare av kredithanteringssystemet CPMS. Jag ansvarar för backlog-hantering, dokumentation och säkerställer att arbetet följer agila principer inom Scrum, i nära samarbete med leverantörer och kollegor.

\vspace{0.5cm}
\textbf{Örnsköldsviks Hamn och Logistik AB - UX/UI-Designer}\\
\normalsize \faCalendar \ Dec. 2023 - Juni 2024 \quad \faMapMarker \ Örnsköldsvik (distans), Sverige

Jag ansvarade för kravhantering och design för ny externwebb och intranät i Sitevision. Genom att arbeta tätt med beställare och kollegor skapade jag lösningar som både uppfyllde tekniska möjligheter och visuella mål, vilket resulterade i nöjda kunder.

\vspace{0.5cm}
\textbf{Örnsköldsviks Kommun - UX/UI-Designer}\\
\normalsize \faCalendar \ Dec. 2023 - Juni 2024 \quad \faMapMarker \ Örnsköldsvik (distans), Sverige

Jag ledde workshops och designade ett kostnadseffektivt intranät anpassat för kommunens anställda, där jag balanserade krav på tillgänglighet och design. Projektet resulterade i en hållbar och användarvänlig lösning inom kommunens budgetramar.

\vspace{0.5cm}
\textbf{Jordbruksverket - UX/UI-Designer}\\
\normalsize \faCalendar \ Mars 2023 - Nov. 2023 \quad \faMapMarker \ Jönköping, Sverige

Jag täckte en akut brist på UX-designers i fyra projekt på Jordbruksverket. Genom att använda user story mapping och skapa prototyper i Axure, Adobe XD och Figma, förbättrade jag användarupplevelsen och säkerställde tillgänglighetsstandarder, vilket uppskattades för sin kvalitet och effektiva leverans.

\vspace{0.5cm}
\textbf{Teledyne FLIR - Lösningsarkitekt}\\
\normalsize \faCalendar \ Dec. 2022 - Mars 2023 \quad \faMapMarker \ USA (distans), Sverige

Som lösningsarkitekt utvärderade jag strategier för Teledyne FLIR:s on-prem-applikation och presenterade en plan för att behålla eller migrera den till en molnlösning. Genom att intervjua nyckelpersoner och analysera tredjepartslösningar levererade jag en tydlig väg framåt, uppskattad för mitt lösningsorienterade arbetssätt.

\vspace{0.5cm}
\textbf{Kammarkollegiet - UX/UI-Designer}\\
\normalsize \faCalendar \ Dec. 2022 - Juni 2023 \quad \faMapMarker \ Karlstad (distans), Sverige

Jag analyserade förstudier och processkartor för Kammarkollegiet och skapade användarvänliga prototyper i Figma. Genom att leda designprocessen och säkerställa kravmöten levererade jag en digital lösning som effektivt mötte verksamhetens och användarnas behov.

\vspace{0.5cm}
\textbf{Hoodin AB - UX-Ingenjör}\\
\normalsize \faCalendar \ Feb. 2021 - Dec. 2022 \quad \faMapMarker \ Malmö, Sverige

Som frontendutvecklare på Hoodin ansvarade jag för AWS-infrastruktur och kravanalys. Genom att uppgradera från Vue 2 till Vue 3, och hantera kunddriven utveckling, förbättrade jag systemets skalbarhet och prestanda, vilket resulterade i nöjda kunder och nya affärsmöjligheter.

\vspace{0.5cm}
\textbf{Inwido AB - UX/UI-Designer \& Fullstackutvecklare}\\
\normalsize \faCalendar \ Nov. 2016 - Feb. 2021 \quad \faMapMarker \ Jönköping, Sverige

Jag spelade en nyckelroll i att modernisera Inwidos IT-system och skapa en mer skalbar plattform genom att migrera systemet till React och .NET Core. Arbetet med ”Konfiguratorn” och ”Bildritaren” förbättrade både produktens noggrannhet och användarupplevelse.

\vspace{0.5cm}
\textbf{NYCE Solutions AB - UX/UI-Designer \& Fullstackutvecklare}\\
\normalsize \faCalendar \ Sept. 2015 - Sept. 2016 \quad \faMapMarker \ Jönköping, Sverige

Jag ansvarade för utveckling av e-handelsplattformar och integrationer med NYCE Logics WMS. Genom workshops och kundintervjuer optimerade jag processer och utvecklade lösningar som förbättrade användarupplevelsen och systemintegrationer för företag som Kjell \& Company, Lideco, Ted Bernhardtz och Bring.

\vspace{0.5cm}

\section*{Kompetenser}

% Section for Expert level
\textbf{5. Expert inom området}

User Story Mapping, UX Design, UX/UI Design, Accessibility (WCAG), System Development, Availibility Testing, HTML, CSS3, Prototyping (Low-Fidelity/High-Fidelity), User Testing and Analysis, User Research, Agile Development (Scrum, Kanban), Process Mapping, Accessibility Testing (Web, Android, iOS), Figma, Adobe Suite (Illustrator, Photoshop, XD), Sketch, HTML5, SCSS/SASS, UML, User-Centered Design \& Product Development, Requirements Analysis, Wireframes / Mockups, Test of Usability (A/B Testing), Microsoft Visio, Responsive Design, Visual Studio Code, User Scenarios, Analysis and Requirements Gathering, Agile Scrum, Wireframing, Persona Development, Journey Mapping, Flowcharting, Interaction Design
\vspace{0.3cm}

% Section for Very High level
\textbf{4. Mycket hög kompetens}

Vue 3, Vue.js, ASP.NET MVC, Entity Framework, VoiceOver, TalkBack, RESTful APIs, JIRA, Trello, Miro, JavaScript, TypeScript, React.js, .Net, .Net Core, C\#, UI Design, Frontend Development, Design Systems, Git, GitHub, DevOps, MySQL, MS SQL, Axure, Web Forms, Design Thinking, draw.io, RESTful API, Project Management, SOAP API, Microsoft SQL Server, Workshops, Material Design, Windows, Business Development, Cross-Browser Compatibility, Version Control (Git), ESLint, Color Theory, Generative AI
\vspace{0.5cm}

% Section for High level
\textbf{3. Hög kompetens}

SVN Tortoise, NPM, Webpack, Three.js, Bootstrap, AWS, Microsoft Azure, Nuxt.js, Angular, Node.js, Express.js, Visual Studio, Tailwind CSS, Content Management Systems, CMS, LINQ, Copywriting, AJAX, JQuery, Android Studio, MAC, Integration, NPM/Yarn, Domain-Driven Development, E-commerce Platforms, Information Architecture, Microinteractions, Sitevision
\vspace{0.5cm}

% Section for Medium level
\textbf{2. Medelkompetens}

Pyramid, PHP 7, SQL Server Management Studio, SAFe (Scaled Agile Framework), Python, Manufacturing Industry, Healthcare Industry, Window Manufacturing, Babel, Logistics \& Warehouse Management Systems (WMS), Microsoft Internet Information Services (IIS), Cypress, Jest, InVision
\vspace{0.5cm}

% Section for Basic level
\textbf{1. Grundläggande kompetens}

Terraform, Linux, Entertainment Industry, GraphQL

\vspace{0.5cm}

\section*{Certifikat}

% Set color and draw box
\setboxcolor
\begin{altbox}[\boxcolor]
    \textbf{Certified Professional in Accessibility Core Competencies} \hfill 
    
    \faCalendar \ Mars 2024 - Mars 2027 \quad \faMapMarker \ International Association of Accessibility Professionals
\end{altbox}

% Education section
\section*{Utbildning}
\setcounter{boxnumber}{0}
% Set color and draw box
\setboxcolor
\begin{altbox}[\boxcolor]
    \textbf{Inbyggda System} \hfill 
    
    \faCalendar \ Aug. 2013 - Juni 2015 \quad \faMapMarker \ Jönköpings Tekniska Högskola, Jönköping
\end{altbox}
\vspace{-5pt}
\setboxcolor
\begin{altbox}[\boxcolor]
    \textbf{Webbutvecklare} \hfill 
    
    \faCalendar \ Aug. 2012 - Juni 2013 \quad \faMapMarker \ Högskolan Väst, Trollhättan
\end{altbox}

\vspace{0.8cm}

% Work experience section
\section*{Arbetsgivare}

\setboxcolor
\begin{altbox}[\boxcolor]
    \textbf{Mercedes-Benz Finans Sverige AB} \hfill 
    
    \faCalendar \ Aug. 2024 - Pågående \quad \faMapMarker \ Malmö, Sverige
\end{altbox}
\vspace{-5pt}
\setboxcolor
\begin{altbox}[\boxcolor]
    \textbf{B3 Grit AB} \hfill 
    
    \faCalendar \ Dec. 2022 - Juni 2024 \quad \faMapMarker \ Malmö, Sverige
\end{altbox}
\vspace{-5pt}
\setboxcolor
\begin{altbox}[\boxcolor]
    \textbf{Hoodin AB} \hfill 
    
    \faCalendar \ Feb. 2021 - Dec. 2022 \quad \faMapMarker \ Malmö, Sverige
\end{altbox}
\vspace{-5pt}
\setboxcolor
\begin{altbox}[\boxcolor]
    \textbf{Consid AB} \hfill 
    
    \faCalendar \ Nov. 2016 - Feb. 2021 \quad \faMapMarker \ Jönköping, Sverige
\end{altbox}
\vspace{-5pt}
\setboxcolor
\begin{altbox}[\boxcolor]
    \textbf{NYCE Solutions AB} \hfill 
    
    \faCalendar \ Sept. 2015 - Aug. 2016 \quad \faMapMarker \ Malmö, Sverige
\end{altbox}

\vspace{0.8cm}
% Footer
\begin{center}
    \textit{För mer information, besök min LinkedIn:}\\
    \texttt{https://www.linkedin.com/in/thommy-westlund-02090650/}
\end{center}

\end{document}